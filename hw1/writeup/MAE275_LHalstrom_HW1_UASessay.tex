
\typeout{}\typeout{If latex fails to find aiaa-tc, read the README file!}
%


\documentclass[]{aiaa-tc}% insert '[draft]' option to show overfull boxes

\usepackage{mathptmx}         %CHANGE FONT TO TIMES NEW ROMAN

\usepackage{amsmath}          % for formula writing (i.e. 'split', etc)
\usepackage{rotate}           %rotate/mirror images
\usepackage{cancel}           %draw lines through math to show "goes to zero"
\usepackage{xfrac}            %allows slated and side fractions
\usepackage{subcaption}       %allows captioning individual subfigures
\usepackage{multicol}         %enable environment with multiple columns
\usepackage[mode=buildnew]{standalone}% requires -shell-escape
  % compile with `pdflatex -shell-escape main` or `xelatex  -shell-escape main`



\usepackage{tikz}             %for creating vector graphics diagrams
\usetikzlibrary{backgrounds}  %put backgrounds behind tikz figures
\usetikzlibrary{calc}         %perform calculations within $$
\usetikzlibrary{positioning}  %position tikz elements using "right of, etc"
\usetikzlibrary{angles}       %label angles between lines with arcs
\usetikzlibrary{quotes}       %Put angle label in quotes
\usetikzlibrary{patterns}     %Patterns to fill shapes with







\usepackage{caption}          %caption graphics ('\captionof')
\usepackage{framed}           %allows shaded text
\usepackage[T1]{fontenc}      %allows escaping specific charaters like '_'


%%%%%%%%%%%%%%%%%%%%%%%%%%%%%%%%%%%%%%%%%%%%%%%%%%%%%%
%HYPERLINKS
\usepackage{hyperref}     %insert hyperlines with '\url' or '\href'
\newcommand\linkcolor{blue} %variable to color url text

%%%%%%%%%%%%%%%%%%%%%%%%%%%%%%%%%%%%%%%%%%%%%%%
%CODE LISTING SYNTAX COLORING
\usepackage{color}     %make custom colors for syntax coloring
\usepackage{listings}  %allows code listings
\usepackage{textcomp}  %allows apostrophes to be straight (unidirectional)
% \usepackage{framed}
% \usepackage{caption}
\usepackage{bm}
\usepackage{tcolorbox}        %use for code listing color definitions
\tcbuselibrary{listings}      %allow color defenitions in code listings
\tcbuselibrary{breakable}     %allow tccolorboxes across page breaks
\tcbuselibrary{most}     %allow tccolorboxes across page breaks

\captionsetup[lstlisting]{font={small,tt}} %setup caption style

%Custom Colors
\definecolor{mygreen}{rgb}{0,0.6,0}
\definecolor{mygray}{rgb}{0.5,0.5,0.5}
\definecolor{mymauve}{rgb}{0.58,0,0.82}

\definecolor{sublimeblack}{HTML}{272822}
\definecolor{sublimered}{HTML}{F92672}
\definecolor{sublimeblue}{HTML}{66D9EF}
\definecolor{sublimeyellow}{HTML}{E6DB74}
\definecolor{sublimegrey}{HTML}{75715E}
\definecolor{sublimegreen}{HTML}{66CC33}
\definecolor{sublimeorange}{HTML}{FD971F}
\definecolor{sublimepurple}{HTML}{AE81FF}


%White or Black background toggle
\newcommand\whiteback{0}
\ifnum\whiteback=1%
  %White Background
  \newcommand\backclr{white}
  \newcommand\txtclr{\color{black}}
  \newcommand\txtupclr{black} %same as text color, used elsewhere
  \newcommand\comclr{\color{sublimegrey}}
  % \newcommand\comclr{\color{mygreen}}
  \newcommand\keyclr{\color{sublimeblue}}
  % \newcommand\keyclr{\color{blue}}
  \newcommand\ndkeyclr{\color{sublimered}}
  \newcommand\strclr{\color{mygreen}}
  % \newcommand\strclr{\color{sublimeyellow}}
  \newcommand\frameon{single} %single-frame box around code

\else
  %Black Background (like SublimeText)
  \newcommand\backclr{sublimeblack}
  \newcommand\txtclr{\color{white}} %text color
  \newcommand\txtupclr{white} %same as text color, used elsewhere
  \newcommand\comclr{\color{sublimegrey}}
  \newcommand\keyclr{\color{sublimeblue}}
  \newcommand\ndkeyclr{\color{sublimered}}
  \newcommand\strclr{\color{sublimeyellow}}
  \newcommand\frameon{false} %no frame for black background
\fi


%Listing Style
% \newcommand\lstfontsize{\scriptsize}
\newcommand\lstfontsize{\small}
\lstdefinestyle{mystyle}{%
  % backgroundcolor=\backclr,   % choose the background color; you must add \usepackage{color} or \usepackage{xcolor}
  basicstyle=\ttfamily\txtclr\lstfontsize, % the SIZE OF THE FONTS that are used for the code
  breakatwhitespace=false,         % sets if automatic breaks should only happen at whitespace
  % breaklines=true,                 % sets automatic line breaking
  captionpos=b,                    % sets the caption-position to bottom
  commentstyle=\comclr,            % comment color
  deletekeywords={...},            % if you want to delete keywords from the given language
  escapeinside={\%*}{*)},          % if you want to add LaTeX within your code
  extendedchars=true,              % lets you use non-ASCII characters; for 8-bits encodings only, does not work with UTF-8
  frame=\frameon,                  % adds a frame around the code
  keepspaces=true,                 % keeps spaces in text, useful for keeping indentation of code (possibly needs columns=flexible)
  columns=flexible,
  otherkeywords={zip,enumerate,True,False,None,...},  %add words to be highlighted
  keywordstyle=\keyclr,            % keyword (e.g. 'print') color
  language=Python,                 % the language of the code
  upquote=true,                    % make apostrophes straight (unidirectional)
  alsoletter={<>=-+*/!},            % to avoid coloring operators when they're not
  ndkeywords={=,+,-,*,**,/,+=,*=,-=,/=,<=,>=,==,!=,<,>,... },  % operator keywords
  ndkeywordstyle=\ndkeyclr,         % style of operator keywords
  numbers=left,                    % where to put the line-numbers; possible values are (none, left, right)
  numbersep=5pt,                   % how far the line-numbers are from the code
  numberstyle=\tiny\color{mygray}, % line-number style: size, color
  rulecolor=\color{black},         % if not set, the frame-color may be changed on line-breaks within not-black text (e.g. comments (green here))
  showspaces=false,                % show spaces everywhere adding particular underscores; it overrides 'showstringspaces'
  showstringspaces=false,          % underline spaces within strings only
  showtabs=false,                  % show tabs within strings adding particular underscores
  stepnumber=1,                    % the step between two line-numbers. If it's 1, each line will be numbered
  stringstyle=\strclr,             % string color
  tabsize=4,                       % sets default tabsize
}


%Function that calls listing of specific file with custom coloring
  %first input is filename to list
  %next two inputs are start and end line numbers
    %(use 0 and large number for all lines)
\newcommand\mylisting[3]{
  \tcbinputlisting{
        listing file=#1,
        breakable,
        nobeforeafter,
        arc=0pt,
        top=0mm,
        bottom=0mm,
        left=0mm,
        right=0mm,
        boxrule=0pt,
        colback=\backclr, %background color (continue accross page breaks)
        colupper=\txtupclr, %basic text color (continue accross page breaks)
        listing only,
        listing options={
                            style=mystyle,            %use custom style
                            firstline=#2,lastline=#3, %lines of code to use
                            firstnumber=#2,           %same numbering as script
                        },
        % hbox %Turning this on prevents page breaks
      }
}







  \title{MAE 275 UAS -- Homework\#1 \\ On The Current State of UAS}


\author{
  Logan D. Halstrom \\
  {\normalsize\itshape Graduate Student} \\
  {\normalsize\itshape Department of Mechanical and Aerospace Engineering} \\
  {\normalsize\itshape University of California, Davis, CA 95616}
       }


 % Define commands to assure consistent treatment throughout document
 \newcommand{\eqnref}[1]{(\ref{#1})}
 \newcommand{\class}[1]{\texttt{#1}}
 \newcommand{\package}[1]{\texttt{#1}}
 \newcommand{\file}[1]{\texttt{#1}}
 \newcommand{\BibTeX}{\textsc{Bib}\TeX}



%%%%%%%%%%%%%%%%%%%%%%%%%%%%%%%%%%%%%%%%%%%%%%%%%%%%%%%%%%%%%%%%%%%%%%%%
\begin{document}

\maketitle


In recent years, there has been a major advent in the development of Unmanned Aerial System (UAS) technology, especially in the civilian and commercial sectors.  Advancements in wireless communication, on-board computing, imaging systems, and power storage have all come together recently to make functional, affordable UAS a reality.  Facing this influx of a new, world-changing technology, there are a few questions that need to be answered.

%%%%%%%%%%%%%%%%%%%%%%%%%%%%%%%%%%%%%%%%%%%%%%%%%%%%%%%%%%%%%%%%%%%%%%%%
\section*{What is most exciting about the future of UAS?}
%%%%%%%%%%%%%%%%%%%%%%%%%%%%%%%%%%%%%%%%%%%%%%%%%%%%%%%%%%%%%%%%%%%%%%%%

UAS are exciting because they hold the potential to not only fill roles that are traditionally held by large, expensive aircraft but also create their own new niches in society.  Automation of flight has the immediately obvious benefit of improved safety to the human that would otherwise have been required to pilot the aerial vehicle in question.  This is especially important for tasks that require particularly dangerous flight operations.  One example is manual inspection of tall structures such as bridges, radio towers, dams, etc.  Typically, these jobs require a human to be tethered to a structure or helicopter and hang hundreds of feet in the air.  This obvious hazard can be eliminated by using UAS equipped with appropriate imaging technology.  The removal of a job hazard, aside from being enough motivation for implementing this technology on its own, could also have the effect of improving the safety of the structure being inspected.  If the inspection task was suddenly made safer, and thus less expensive, it could be performed more often or more thoroughly than previously, ensuring better confidence after an inspection.

UAS also have the potential to improve the safety of police officers and soldiers in the line of duty.  It is the nature of these occupations to deal with the unknown, which can prove deadly when the wrong decision is made.  A surveillance UAS provides the ability to learn more about the unknown environment without endangering a human life.  Thus, police officers and soldiers would be able to make more informed decisions when in a dangerous or hostile environment.

Another advantage of the automation and size reduction available in a UAS is the cost reduction of a task typically performed by a full-size aircraft.  These tasks range from crop dusting to air science to aerial surveillance for search-and-rescue or journalism.  As before, automating these processes also holds the additional benefit of improving their safety by removing the human pilot.

As mentioned above, UAS also hold the potential to perform new and unique jobs.  By removing the pilot, the human limitations on flight (sleep, eating, etc.) no longer restrain the vehicle, allowing for single missions that extend much longer than was previously possible.  This opens the door for all kinds of new ideas for aircraft, like real time imaging of the earth, power generation, or wireless communication.

UAS also have many implications in warfare and may create new tactics.  UAS swarms are a unique kind of force that could accomplish tasks varying from sensor jamming or spoofing to wide-spread surveillance, and may even hold abilities that have yet to be discovered.

This is nowhere near a complete list of the potential advantages of UAS use in the future, but rather than continue to laud the benefits of this technology, it is also necessary to address the potentially negative implications as well.

\clearpage
%%%%%%%%%%%%%%%%%%%%%%%%%%%%%%%%%%%%%%%%%%%%%%%%%%%%%%%%%%%%%%%%%%%%%%%%
\section*{What is most disturbing about the future of UAS?}
%%%%%%%%%%%%%%%%%%%%%%%%%%%%%%%%%%%%%%%%%%%%%%%%%%%%%%%%%%%%%%%%%%%%%%%%

Above, it was stated that UAS will create their own niches in our society.  The positive aspect of this is that new benefits may be unlocked that were not available to humankind before, but the negative aspect is that the exact same is true for undesirable effects.

One of the most apparent potential issues with UAS is the effect of increasing aerial surveillance of the populace.  UAS altitude allows imaging of areas that have been previously considered as "private" and future UAS may be so small that they could enter private areas unseen and act as a "fly-on-the-wall", so to speak.  It is technologies like these that make an Orwellian future seem a tangible possibility.

Another UAS issue already in effect is the impact on aviation safety.  While some applications of UAS improve human safety (as stated in the previous section), the interaction of small UAS and large, crewed aircraft is a new phenomenon that does not have an existing system to safely govern their interaction.  There have already been incidents of UAS near-misses and collisions with manned air vehicles, so the threat of a deadly accident is very real.

A yet to be seen outcome of UAS adoption could be the loss of certain jobs to automation.  The previous section described a number of jobs that could be replaced by UAS to improve human safety, but "removing the human" implies that their job, often a skilled trade, will be made null and void.

Finally, a less critical but still important risk of increased dependency on UAS technology is an increased dependency on GPS by extension.  GPS has already become an integral part of many industries and represents a single failure point in our infrastructure that could cause a massive societal disruption if disabled.  Automation through UAS would only expedite our dependency on this technology.





%%%%%%%%%%%%%%%%%%%%%%%%%%%%%%%%%%%%%%%%%%%%%%%%%%%%%%%%%%%%%%%%%%%%%%%%
\section*{What are some of the main hurdles in future of UAS?}
%%%%%%%%%%%%%%%%%%%%%%%%%%%%%%%%%%%%%%%%%%%%%%%%%%%%%%%%%%%%%%%%%%%%%%%%

The biggest impediment on UAS progress will be legislation over their usage.  Current operating limitations enacted by the FAA are extremely restrictive of the abilities of UAS, outlawing commercial use and making obvious applications of the technology like deliveries impossible due to the line-of-sight requirement.  These regulations will soon be relaxed, but this will only be the beginning of extensive regulation required for integrating UAS into our already matured systems for airspace control and individual rights.  Aside from legislation, court cases will undoubtedly occur that will shape the usage of UAS through experience.

Individual privacy is a right protected in the Constitution of the United States, but the nature of what that privacy pertains to is a contested subject that is finely adjusted by various laws and court cases.  Aside from the inherent "wrongness" of the invasion of privacy, this nature of UAS also creates a negative public view of "drones", which creates a resistance to the acceptance of the technology.

Stepping aside from policy, there are also still technological limitations that must be addressed to make some UAS industries viable.  Battery technology is of primary concern, as it is directly related to a UAS's payload capacity and range.  As of now, the state of battery technology has come leaps and bounds from that of a few decades ago, but even more improvement is required for the future.

%%%%%%%%%%%%%%%%%%%%%%%%%%%%%%%%%%%%%%%%%%%%%%%%%%%%%%%%%%%%%%%%%%%%%%%%
\section*{How can UAS be made to be beneficial to humankind?}
%%%%%%%%%%%%%%%%%%%%%%%%%%%%%%%%%%%%%%%%%%%%%%%%%%%%%%%%%%%%%%%%%%%%%%%%

UAS technology is ripe with both apparent benefits and risks to modern society.  To reap these benefits of UAS responsibly without compromising our individual rights and safety, clear, well thought out legislation and policy will be required for the development, operation, and usage of UAS.  The safety of our airspace must be maintained, so new aircraft regulations will need to be introduced by the FAA and new forms of Air Traffic Control that incorporate both manned and unmanned vehicles will need to be adopted.

Aside from safety, legislation governing privacy will need to be updated to ensure individuals are just as protected as before from other individuals and the government.  However, no amount of regulation will be enough to make things exactly as they were before, so there will need to be some cultural adjustment to the definition of "privacy".  New technologies like cellphone cameras have already shown that this adjustment can be made.







\end{document}


